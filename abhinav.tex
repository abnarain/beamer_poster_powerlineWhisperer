% fancytikzposter.tex, version 2.1
% Original template created by Elena Botoeva [botoeva@inf.unibz.it], June 2012
% 
% This file is distributed under the Creative Commons Attribution-NonCommercial 2.0
% Generic (CC BY-NC 2.0) license
% http://creativecommons.org/licenses/by-nc/2.0/ 


\documentclass{a0poster}

\usepackage{fancytikzposter} 


%%%%% --------- Change here if you want ---------- %%%%%
%% margin for the geometry package, must be changed before using the geometry package
%% default value is 4cm
% \setmargin{4}

%% the space between the blocks
%% default value is 2cm
% \setblockspacing{2}

%% the height of the title stripe in block nodes, decrease it to save space
%% default value is 3cm
% \setblocktitleheight{3}

%% the number of columns in the poster, possible values 2,3
%% default value is 2
% \setcolumnnumber{3}

%% the space between two or more groups of authors from different institutions
%% used in \maketitle
% \setinstituteshift{10}

%% which template to use
%% N1 simple, standard look, with a colored background and gray boxes
%% N2 board with nodes
%% N3 another standard look
%% N4 envelope-like look
%% N5 with a wave-like head, original idea taken from
%%%% http://fc09.deviantart.net/fs71/f/2010/322/1/1/scientific_poster_by_nabuy-d333ria.jpg
%\usetemplate{6}

%% components of the templates
%% (the maximal possible numbers are mentioned as the parameters)
% \usecolortemplate{4}
% \usebackgroundtemplate{5}
% \usetitletemplate{2}
% \useblocknodetemplate{5}
% \useplainblocktemplate{4}
% \useinnerblocktemplate{2}


%% the height of the head drawing on top 
%% applicable to templates N3, 4 and 5
% \setheaddrawingheight{14}


%% change the basic colors
%\definecolor{myblue}{HTML}{008888} 
%\setfirstcolor{myblue}% default 116699
%\setsecondcolor{gray!80!}% default CCCCCC
%\setthirdcolor{red!80!black}% default 991111

%% change the more specific colors
% \setbackgrounddarkcolor{colorone!70!black}
% \setbackgroundlightcolor{colorone!70!}
% \settitletextcolor{textcolor}
% \settitlefillcolor{white}
% \settitledrawcolor{colortwo}
% \setblocktextcolor{textcolor}
% \setblockfillcolor{white}
% \setblocktitletextcolor{colorone}
% \setblocktitlefillcolor{colortwo} %the color of the border
% \setplainblocktextcolor{textcolor}
% \setplainblockfillcolor{colorthree!40!}
% \setplainblocktitletextcolor{textcolor}
% \setplainblocktitlefillcolor{colorthree!60!}
% \setinnerblocktextcolor{textcolor}
% \setinnerblockfillcolor{white}
% \setinnerblocktitletextcolor{white}
% \setinnerblocktitlefillcolor{colorthree}


%%% size of the document and the margins
%% A0
% \usepackage[margin=\margin cm, paperwidth=118.9cm, paperheight=84.1cm]{geometry} 
\usepackage[margin=\margin cm, paperwidth=30in, paperheight=40in]{geometry}
%% B1
% \usepackage[margin=\margin cm, paperwidth=70cm, paperheight=100cm]{geometry}

%% changing the fonts
\usepackage{cmbright}
%\usepackage[default]{cantarell}
%\usepackage{avant}
%\usepackage[math]{iwona}
\usepackage[math]{kurier}
\usepackage[T1]{fontenc}
\usepackage{graphicx}


%% add your packages here
\usepackage{hyperref}

\title{PowerlineWhisperer}
\author{*Abhinav Narain, Nick Feamster, Mung Chiang, *Matthieu Bloch\\
  Princeton University, *Georgia Tech\\
  \texttt{ anarain@cs.princeton.edu ,feamster@cs.princeton.edu, } \\
  \texttt{mchiang@princeton.edu, bloch@ece.gatech.edu}
}

\begin{document}
%%%%% ---------- the background picture ---------- %%%%%
%% to change it modify the macro \BackgroundPicture
\ClearShipoutPicture
\AddToShipoutPicture{\BackgroundPicture}

\noindent % to have the picture right in the center
\begin{tikzpicture}
  \initializesizeandshifts
  % \setxshift{15}
  % \setyshift{2}
  %% the title block, #1 - shift, the default value is (0,0), #2 - width, #3 - scale
  %% the alias of the title block is `title', so we can refer to its boundaries later
  \ifthenelse{\equal{\template}{1}}{ 
    \titleblock{47}{1}
  }{
    \titleblock{47}{1.5}
  }

  %% a logo can be added to the title block
  %% #1 - anchor relative to the title block, #2 - shift, #3 - width, #3 - file name
  \ifthenelse{\equal{\template}{1}}{ 
   %  \addlogo[south east]{(2,0)}{6cm}{figures/GeorgiaTechSeal.png}
   %  \addlogo[south west]{(2,0)}{6cm}{figures/princetonLogo.png}
  }{
     %\addlogo[south east]{(2,0)}{6cm}{figures/GeorgiaTechSeal.png}
  }


  %% by default, the position of the new block node is right below the previous
  %% block node, stored in (currenty)
  %% box is the alias of the previous block, so we can refer to its boundaries

  %%%%%%%%%% ------------------------------------------ %%%%%%%%%%
  \blocknode{Problem Statement}%
  {
    \begin{center}
    \includegraphics[scale=0.85]{figures/setup.pdf} 
    \end{center}

    \begin{center}
    \includegraphics[scale=0.50]{figures/emi.png}
    \end{center}

  }

  \blocknode{Theoretical Framework}%
  {
    \begin{center}
    \includegraphics[scale=1.0]{figures/model.pdf} 
    \end{center}

\coloredbox{colorthree!50!}{ 
    Let the adversary use a hypothesis detector with the following two
    hypothesis:
    \begin{itemize}
    \item \textbf{$H_{0}$}: Covert communication is \textit{not} happening.
    \item \textbf{$H_{1}$}: Covert communication is  happening.
    \end{itemize}

    Willie's optimal hypothesis test yields the tradeoff $\alpha + \beta \geq 1 -
    V(Q_{0}^{n},\hat{Q_{n}})$, where
$V(Q_{0}^{n},\hat{Q_{n}})$ is the variational distance between the true
distribution (that is of noise) represented by $Q_{0}^{n}$ and estimated
distribution $\hat{Q_{n}}$, which is in the presence of
communication.
}


    \begin{center}
    \includegraphics[scale=0.85]{figures/noise.pdf} 
    \end{center}


\coloredbox{colorthree!50!}{ 
Adversary capability :
Monitoring device plugged into powerline network
Measurement device at Mains switch of the setup.
}





  }

  %%%%%%%%%%%%% NEW COLUMN %%%%%%%%%%%%%%% 
  \startsecondcolumn 

  %%%%%%%%%% ------------------------------------------ %%%%%%%%%%
  \blocknode%
  {System Design }
  {

    Transmit chain at the transmitter consists of several blocks which convert
    the original message to waveforms on Powerline 
    \begin{center}
    \includegraphics[scale=1.0]{figures/transmitter.pdf} \\
    \end{center}
    Receive chain at the receiver consists of 
    \begin{center}
    \includegraphics[scale=1.0]{figures/receiver.pdf}
    \end{center}

    \begin{center}
    \includegraphics[scale=0.85]{figures/schematic.pdf} \\
    \end{center}
    \begin{center}
    \includegraphics[scale=0.20]{figures/setup_together.png} \\
    \end{center}

  }

  \blocknode{Results}%
  { 
    \begin{center}
    \includegraphics[scale=0.45]{figures/roc_curve.pdf} \\
    \end{center}

    \begin{center}
    \includegraphics[scale=0.45]{figures/ncurve_53_1_05_iq.pdf}
    \end{center}
  }
  %%%%%%%%%% ------------------------------------------ %%%%%%%%%%


\end{tikzpicture}
\end{document}
