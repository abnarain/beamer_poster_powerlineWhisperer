% fancytikzposter.tex, version 2.1
% Original template created by Elena Botoeva [botoeva@inf.unibz.it], June 2012
% 
% This file is distributed under the Creative Commons Attribution-NonCommercial 2.0
% Generic (CC BY-NC 2.0) license
% http://creativecommons.org/licenses/by-nc/2.0/ 


\documentclass{a0poster}

\usepackage{fancytikzposter} 


%%%%% --------- Change here if you want ---------- %%%%%
%% margin for the geometry package, must be changed before using the geometry package
%% default value is 4cm
% \setmargin{4}

%% the space between the blocks
%% default value is 2cm
% \setblockspacing{2}

%% the height of the title stripe in block nodes, decrease it to save space
%% default value is 3cm
% \setblocktitleheight{3}

%% the number of columns in the poster, possible values 2,3
%% default value is 2
% \setcolumnnumber{3}

%% the space between two or more groups of authors from different institutions
%% used in \maketitle
% \setinstituteshift{10}

%% which template to use
%% N1 simple, standard look, with a colored background and gray boxes
%% N2 board with nodes
%% N3 another standard look
%% N4 envelope-like look
%% N5 with a wave-like head, original idea taken from
%%%% http://fc09.deviantart.net/fs71/f/2010/322/1/1/scientific_poster_by_nabuy-d333ria.jpg
%\usetemplate{6}

%% components of the templates
%% (the maximal possible numbers are mentioned as the parameters)
% \usecolortemplate{4}
% \usebackgroundtemplate{5}
% \usetitletemplate{2}
% \useblocknodetemplate{5}
% \useplainblocktemplate{4}
% \useinnerblocktemplate{2}


%% the height of the head drawing on top 
%% applicable to templates N3, 4 and 5
% \setheaddrawingheight{14}


%% change the basic colors
%\definecolor{myblue}{HTML}{008888} 
%\setfirstcolor{myblue}% default 116699
%\setsecondcolor{gray!80!}% default CCCCCC
%\setthirdcolor{red!80!black}% default 991111

%% change the more specific colors
% \setbackgrounddarkcolor{colorone!70!black}
% \setbackgroundlightcolor{colorone!70!}
% \settitletextcolor{textcolor}
% \settitlefillcolor{white}
% \settitledrawcolor{colortwo}
% \setblocktextcolor{textcolor}
% \setblockfillcolor{white}
% \setblocktitletextcolor{colorone}
% \setblocktitlefillcolor{colortwo} %the color of the border
% \setplainblocktextcolor{textcolor}
% \setplainblockfillcolor{colorthree!40!}
% \setplainblocktitletextcolor{textcolor}
% \setplainblocktitlefillcolor{colorthree!60!}
% \setinnerblocktextcolor{textcolor}
% \setinnerblockfillcolor{white}
% \setinnerblocktitletextcolor{white}
% \setinnerblocktitlefillcolor{colorthree}


%%% size of the document and the margins
%% A0
% \usepackage[margin=\margin cm, paperwidth=118.9cm, paperheight=84.1cm]{geometry} 
\usepackage[margin=\margin cm, paperwidth=30in, paperheight=40in]{geometry}
%% B1
% \usepackage[margin=\margin cm, paperwidth=70cm, paperheight=100cm]{geometry}

%% changing the fonts
\usepackage{cmbright}
%\usepackage[default]{cantarell}
%\usepackage{avant}
%\usepackage[math]{iwona}
\usepackage[math]{kurier}
\usepackage[T1]{fontenc}
\usepackage{graphicx}


%% add your packages here
\usepackage{hyperref}
%\usepackage{algorithm}
%\usepackage{ algorithmic}
\usepackage{algpseudocode}
\usepackage{caption}

\title{PowerlineWhisperer}
\author{*Abhinav Narain, Nick Feamster, Mung Chiang, *Matthieu Bloch\\
  Princeton University, *Georgia Tech\\
  \texttt{ anarain@cs.princeton.edu ,feamster@cs.princeton.edu, } \\
  \texttt{mchiang@princeton.edu, bloch@ece.gatech.edu}
}

\begin{document}
%%%%% ---------- the background picture ---------- %%%%%
%% to change it modify the macro \BackgroundPicture
\ClearShipoutPicture
\AddToShipoutPicture{\BackgroundPicture}

\noindent % to have the picture right in the center
\begin{tikzpicture}
  \initializesizeandshifts
  % \setxshift{15}
  % \setyshift{2}
  %% the title block, #1 - shift, the default value is (0,0), #2 - width, #3 - scale
  %% the alias of the title block is `title', so we can refer to its boundaries later
  \ifthenelse{\equal{\template}{1}}{ 
    \titleblock{47}{1}
  }{
    \titleblock{47}{1.5}
  }

  %% a logo can be added to the title block
  %% #1 - anchor relative to the title block, #2 - shift, #3 - width, #3 - file name
  \ifthenelse{\equal{\template}{1}}{ 
   %  \addlogo[south east]{(2,0)}{6cm}{figures/GeorgiaTechSeal.png}
   %  \addlogo[south west]{(2,0)}{6cm}{figures/princetonLogo.png}
  }{
     %\addlogo[south east]{(2,0)}{6cm}{figures/GeorgiaTechSeal.png}
  }


  %% by default, the position of the new block node is right below the previous
  %% block node, stored in (currenty)
  %% box is the alias of the previous block, so we can refer to its boundaries

  %%%%%%%%%% ------------------------------------------ %%%%%%%%%%
  \blocknode{Problem Statement}%
    {
     Digital communication is more vulnerable to eavesdropping as it gets easier to intercept without detection with low cost and efficient storage. 

    \vspace{0.3cm}
    \textbf{Purpose} 
    Enable deniable communication between two parties in
    close proximity in presence of passive monitoring. 
    \begin{center}
    \includegraphics[scale=0.85]{figures/setup.pdf} 
    \captionof{figure}{Application scenario- Powerline can be used to covertly bridge Alice and Bob on different VLANs in an
    enterprise network. Willie is an adversary interested in detecting covert communication over powerline. }
    \end{center}

    \vspace{0.3cm}
    \textbf{Approach} Leverage ubiquitous power transmission infrastructure
    at our disposal and the ubiquitous presence of noise  which can be used as
    cover for physical layer covert communication. 
    \begin{center}
    \includegraphics[scale=0.20]{figures/emi.png}
    \captionof{figure}{Presence of noise due to various appliances plugged in powerline}
    \end{center}
    
   }

  \blocknode{Threat Model}%
  {
    We represent a real scenario by modelling the two parties Alice and Bob, doing covert
    communication in the presence of an adversary Willie. Alice encodes the
    message with a secret key which decides the timeslot of transmission. The
    message is received at Bob and Willie over the channel $W_{Y|X}$,
    $W_{Y|Z}$ respectively. The distribution of amplitude of the encoded message
    observed at Alice is $X$, Bob is $Y$ and at Willie is $Z$.
    \begin{center}
    \includegraphics[scale=1.5]{figures/model.pdf} 
    \captionof{figure}{Information-theoretic modeling}
    \end{center}

    \vspace{0.3cm}
    \textbf{Adversary capability}
    \begin{itemize}
      \item Monitoring device to measure quadrature samples at Mains supply of
      physical powerline network 
      \item Unlimited storage capability and ability to do analysis of collected data
    \end{itemize}
    Let the adversary use a hypothesis detector with the following two hypothesis:
    \begin{itemize}
    \item \textbf{$H_{0}$}: Covert communication is \textit{not} happening.
    \item \textbf{$H_{1}$}: Covert communication is  happening.
    \end{itemize}

    Willie's optimal hypothesis test yields the tradeoff \\
    $\alpha + \beta \geq 1 - V(Q_{0}^{n},\hat{Q_{n}}) $ ,
    where $V(Q_{0}^{n},\hat{Q_{n}})$ is the variational distance between the true
    distribution (that is of noise) represented by $Q_{0}^{n}$ and estimated
    distribution $\hat{Q_{n}}$, which is in the presence of communication.
    $\alpha$, $\beta$ represent the probability of type I and type II errors.

    \begin{center}
    \includegraphics[scale=0.35]{figures/noise.pdf}     
    \captionof{figure}{Basis of deniable communication. Distribution of
    noise floor in presence of deniable communication(red) should
    mimic the noise floor(blue) and have high overlap for low detec-
    tion by adversary.}

    \end{center}
    
   }
  %%%%%%%%%%%%% NEW COLUMN %%%%%%%%%%%%%%% 
  \startsecondcolumn 

  %%%%%%%%%% ------------------------------------------ %%%%%%%%%%
  \blocknode{System Design }% 
  {
      PowerlineWhisperer provides different degrees of freedom to be exercised
      by the user for covert communicate over powerline 
      \begin{itemize}
        \setlength\itemsep{0em}
      \item Frequency of transmission
      \item Bandwidth of the channel
      \item Transmission power
      \item Message size for transmission
      \item Pulse shape
    \end{itemize}

    \vspace{0.3cm}
    \coloredbox{colorthree!80!}{
    \textbf{Transmission Strategy}
    %\begin{algorithm}  [H]
    \begin{algorithmic}[1]
      \Procedure{Transmission Slot\textendash Selection}{}\newline
       Initialize the random number generator with the secret key. Let the
       message size be $M$ in number of bits. 
      \For{$1$ \ldots  $M$ } 
         \State {Choose a number from a uniform distribution ($0$,$M^{2}$)
              generated by the random number generator} 
         \State {Place your message bit in the slot}
         \EndFor
         \EndProcedure
            \end{algorithmic}
            Transmit the whole sequence of bits on the channel 
     %\end{algorithm}
    }
    \vspace{0.3cm}
    \textbf{Transmitter and Receiver Block Chain}
    \begin{itemize}
      \item PowerlineWhisperer implements various blocks in software define
      radio (Gnuradio) for transmitter and receiver
      \item Receiver blocks convert the signals from powerline to original message using the algorithm described
      \end{itemize}
    \begin{center}
    \includegraphics[scale=1.0]{figures/transmitter.pdf} 
      \captionof{figure}{Transmitter Block Chain}
    \end{center}
    \begin{center}
    \includegraphics[scale=1.0]{figures/receiver.pdf}
      \captionof{figure}{Receiver Block Chain}
    \end{center}
    
  }
  %%%%%%%%%% ------------------------------------------ %%%%%%%%%%

  \blocknode{Results}%
  { 
    We evaluate the performance of \textit{PowerlineWhisperer} by measuring the
    change in the distribution in the amplitude of quadrature sample   
   The three subplots show normalized histograms of the amplitude of powerline
   channel at two extreme situations of presence and absence of transmission.
   Channel profile 
   \begin{itemize}
    \item In absence of transmissions (channel noise)
    \item In the presence of covert transmissions of message of 112 bytes operating at 2 MHz with 100 KHz bandwidth
    \item Continuous transmission of pulses
    \item Scheme achieves a throughput of 28 bps. OFDM can improve the
    throughput significantly
    \item The scheme has a detection rate of $0.57$, which is almost as good as
    flip of a coin
    \end{itemize}
    \begin{center}
    \includegraphics[scale=0.50]{figures/ncurve_53_1_05_iq.pdf}
    \end{center}

    %\abegin{center}
    %\includegraphics[scale=0.25]{figures/roc_curve.pdf} 
    %  \captionof{figure}{ ROC curve at the adversary. This shows that a energy detector with a threshold rule is unable to distinguish
    %      the operation of covert communication as the curve is close to $1 −\beta = \alpha$ line. The area under the curve is 0.5715311 }
    %\end{center}
  }
  %%%%%%%%%% ------------------------------------------ %%%%%%%%%%

\end{tikzpicture}
\end{document}
